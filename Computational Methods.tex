\documentclass[10pt,a4paper]{scrartcl}

\usepackage[english]{babel}

\input{../Headerfiles/Packages}
\input{../Headerfiles/Titles}
\input{../Headerfiles/Commands}

\renewcommand{\baselinestretch}{1.3}
\parindent 0pt

\author{GianAndrea Müller / Pascal Müller}
\title{CMEA II}

\begin{document}

\begin{multicols*}{3}
\maketitle
\small
\tableofcontents
\normalsize
\end{multicols*}

\newpage

\begin{multicols*}{3}

\section{ODE}

Differential equation: $\mathbb{F}\left(t,u,u_t,u_{tt},\ldots\right)=0$

\subsection{Definitions}

\begin{tabular}{ll}
Autonomous $F(U(t))$ &Non-Autonomous $F(U(t),t)$\\
F does not depend on t& $u'(t)=u(t)+t$\\
\hline
Linear & Non-Linear\\
F is linear in $u(t)$&$u'(t)=\sin(u(t))$
\end{tabular}

\subsection{IVP}

To find a specific solution to an ODE of degree $k$ we need $(k-1)$ initial values.

$\mathbb{F}\left(t,u,u',u'',u^{(3)},\ldots,u^{(k)}\right)=0$ \hfill with\hfill $u^{(i)}(0)=u_{i}\quad\forall i$

\subsubsection{Single Body Dynamics}

\begin{tabular}{ll}
$x'(t)=v(t),$&$x(0)=x_0$\\
$v'(t)=f,$&$v(0)=v_0$
\end{tabular}

\begin{center}$U=[x,v],\qquad F=[v,f]$

$U'=F(t,U),\qquad U(0)=[x_0,v_0]$
\end{center}

\subsubsection{Trick: Vectorization}

begin with $\theta''(t)=-\frac{g}{L}\sin(\theta(t))$

\small
$\vec{u}(t)=\begin{pmatrix}\theta(t)\\\theta'(t)\end{pmatrix}\quad\vec{u}'(t)=\begin{pmatrix}\theta'(t)\\\theta''(t)\end{pmatrix}=\begin{pmatrix}\theta'(t)\\-\sin(\theta(t))\end{pmatrix}=\vec{F}(t,\vec{u}(t))$\normalsize

\subsubsection{N-Body Dynamics}

\begin{tabular}{ll}
$x_i'=v_i,$&$i=1,2,\ldots,N,$\\
$m_iv_i'=f_i,$&$i=1,2,\ldots,N,$
\end{tabular}

$f_i=\sum\limits_{i=1,j\neq 1}^N\frac{Gm_im_j(x_i-x_j)}{\left|x_i-x_j\right|^3}\qquad$ \note{G - Gravitational constant}

\begin{center}
$U=[x_1,\ldots,x_N,v_1,\ldots,v_n]$

$F=[v_1,\ldots,v_n,f_1,\ldots,f_N]$

$U'(t)=F(U(t)),\qquad U(0)=[x_1^0,\ldots,x_N^0,v_1^0,\ldots,v_N^0]$
\end{center}

\subsection{Matrix exponential technique}

\mportabflex{l@{ = }l}{
$u'(t)$&$Au(t)$\\
$u(0)$&$u_0$
}

Diagonalise $A=R\Lambda R^{-1}$

Then $(R^{-1}u(t))'=R^{-1}(R\Lambda R^{-1})u(t)=\Lambda R^{-1}u(t)$

Now let $w=R^{-1}u$

\mportabflex{l@{ = }l}{
$w'(t)$&$\Lambda w(t)$\\
$w(0)$&$R^{-1}u_0$
}

Thus $w_i(t)=w_i(0)e^{\lambda_i t}$ and $w(t)=e^{\Lambda t}w(0)=e^{\Lambda t}R^{-1}u_0$.

\importabflex{l@{ = }l}{
$R^{-1}u(t)$&$e^{\Lambda t}R^{-1}u_0$\\
$u(t)$&$Re^{\Lambda t}R^{-1}u_0=e^{At}u_0$
}

\important{$e^{At}=I+At+\onha A^2t^2+\cdots+\frac{1}{k!}A^kt^k+\cdots$}

\section{Numerical Methods for ODE}

\begin{align*}
u'(t)&=F(t,u(t)),\\
u(0)&=u_0
\end{align*}

\note{$u_0\in\mathbb{R}^m$ is a constant, $u:[0,T]\rightarrow\mathbb{R}^m$ and $F:[0,T]\times\mathbb{R}^m\rightarrow\mathbb{R}^m$}

\subsubsection{Time discretization}

$[0,T)=\cup_{n=0}^{N-1}[t^n,t^{n+1})$ where $t^n=n\Delta t$ and $\Delta t = \frac{T}{N}$

Approximation of the solution: $\{U_n\}^N_{n=0}\qquad U_n\approx u(t^n)$

\subsection{Forward Euler Method (Explicit)}

\begin{center}
$u'(t^n)\approx\frac{u(t^{n+1})-u(t^n)}{\Delta t}\approx\frac{U_{n+1}-U_n}{\Delta t}$
\end{center}

\note{Good approximation if $\Delta t$ is small}

$u'(t)=F(t,u(t)) \Longrightarrow \frac{U_{n+1}-U_n}{\Delta t}=F(t^n,U^n)$

Solving for $U^{n+1}$ yields:

\important{$U_{n+1}=U_n+\Delta t\ F(t^n,U_n)\qquad U_0=u_0$}

\textbf{Algorithm:}
\begin{algorithmic}
\State $U_0=u_0$
\For {$(n=0;n<N;n++)$}
\State $U_{n+1}=U_{n}+\Delta t\ F(t^n,U_n)$
\EndFor
\end{algorithmic}

\subsection{Backward Euler Method (Implicit)}

\begin{center}
$u'(t^{n+1})\approx\frac{u(t^{n+1})-u(t^n)}{\Delta t}\approx\frac{U_{n+1}-U_n}{\Delta t}$
\end{center}

\important{$U_{n+1}-\Delta t F(t^{n+1},U_{n+1})=U_n\qquad U_0=u_0$}

\note{Solving for $U_{n+1}$ we notice that the result still depends on $F(t^{n+1},U_{n+1})$!

\textbf{Advantage to Forward Euler:} Stability. \textbf{Disadvantage:} More computation necessary.}

\subsection{Trapezoidal Method}

\importname{Fundamental Theorem of Calculus}{$u(t^{n+1})-u(t^n)=\int_{t_n}^{t^{n+1}}u'(s)ds=\int_{t^n}^{t^{n+1}}F(s,u(s))ds$}

\importname{Trapezoidal rule}{$\int_a^bf(x)dx\approx(b-a)\left(\frac{f(a)+f(b)}{2}\right)$}

$\frac{U_{n+1}-U_n}{\Delta t}=\frac{1}{2}\left(F(t^n,U_n)+F(t^{n+1},U_{n+1})\right)$

\important{$U_{n+1}=u_n+\frac{\Delta t}{2}F(t^n,U_n)+\frac{\Delta t}{2}F(t^{n+1},U_{n+1})$}

\subsection{Mid-point Rule}

\begin{center}
$u'(t^n)\approx\frac{u(t^{n+1})-u(t^{n-1})}{2\Delta t}\approx\frac{U_{n+1}-U_{n-1}}{2\Delta t}$
\end{center}

$\frac{U_{n+1}-U_{n-1}}{2\Delta t}=F(t^n,U_n)$

\important{$U_{n+1}=U_{n-1}+2\Delta t F(t^n,U_n)\qquad U_0=u_0$}

\note{In the beginning we only have one initial value, but need two. Possible solution: $U_1=u_0+\Delta F(0,u_0)$ (Forward Euler method)}

\subsection{Trunctuation error}

The trunctuation error is the error made when the exact solution is inserted in the consisten form of numerical methods.

\subsubsection{Example: Forward Euler method}

$T_n=\frac{u(t^{n+1})-u(t^n)}{\Delta t}-F(t^n,u(t^n))$

Taylor:

$\frac{u(t^n)-u(t^n)+\Delta t u'(t^n)+\frac{(\Delta t)^2}{2}u''(t^n)+\mathbb{O}((\Delta t)^3)}{\Delta t}-F(t^n,u(t^n))$

$\underbrace{u't(t^n)-F(t^n,u(t^n))}_{=0 \text{ using IVP}}+\frac{\Delta t}{2}u''(t^n)+\mathbb{O}((\Delta t)^2)=\mathbb{O}(\Delta t)$

\finn

\emph{Therefore the truncation error is of} $\mathcal{O}(\Delta t)$.

\finn

\note{$T_n^{Midpoint}=\mathcal{O}(\Delta t^2)=T_n^{trapezoidal}$

$T_n^{Backward Euler}=\mathcal{O}(\Delta t)$
}

\subsubsection{One-Step Error}

The one step error, is the error discovered when inserting the exact solution into the update for a given method.

\subsubsection{Global Error}

\important{$E_N:=u(t^N)-U_N\leq \sum\limits_{j=0}^{N-1}|L_j|$}

\important{$E_N\approx \mathcal{O}((\Delta t)^{q+1}\cdot N=\mathcal{O}((\Delta t)^{q+1}\cdot\frac{T}{\Delta t}\approx \mathcal{O}((\Delta t)^q)$}

\subsection{Newtons Method}

\mportabflex{l@{ = }l}{
$\vec{G}(x)$&$0$\\
$D\vec{G}(x_k)\Delta x_k$&$-\vec{G}(x_k)$\\
$x_{k+1}$&$x_k+\Delta x_k$
}

\section{Higher-Order Methods for ODEs}

\subsection{Runga-Kutta-2 (RK-2)}

\note{RK-\glqq 2\grqq \dahe 2-stage method}

\mportabflex{r@{ = }l}{$u'(t)$&$F(t,u(t))$\\$u(0)$&$u_0$}

Use the mid-point rule, approximating $u(t^n+\frac{\Delta t}{2})$ with Forward-Euler.

\importabflex{r@{ = }l}{
$Y_1$&$U_n$\\
$Y_2$&$U_n+\frac{\Delta t}{2}F(t^n,Y_1)$\\
$U_{n+1}$&$U_n+\Delta tF(t^n+\frac{\Delta t}{2},Y_2)$\\
$U_0$&$u_0$}

in the update-form:

\importabflex{r@{ = }l}{
$U_{n+1}$&$U_n+\Delta tF(t^n+\frac{\Delta t}{2},U_n+\frac{\Delta t}{2}F(t^n,U_n))$\\
$U_0$&$u_0$
}

\subsubsection{Order of Accuracy}

\begin{enumerate}
\item missing
\end{enumerate}

UNFINISHED - SEE SCRIPT

\subsection{Runge-Kutta-4 (RK-4)}

\importabflex{rcl}{
$Y_1$&=&$U_n$\\
$Y_2$&=&$U_n+\frac{\Delta t}{2}F(t^n,Y_1)$\\
$Y_3$&=&$U_n+\frac{\Delta t}{2}F(t^n+\frac{\Delta t}{2},Y_2$\\
$Y_4$&=&$U_n+\frac{\Delta t}{2}F(t^n+\frac{\Delta t}{2},Y_3$\\
$U_{n+1}$&=&$U_n+\frac{\Delta t}{6}(F(t^n,Y_1)+2F(t^n+\frac{\Delta t}{2},Y_2)$\\
&&$+2F(t^n+\frac{\Delta t}{2},Y_3)+F(t^n+\Delta t,Y_4))$\\
$U_0$&=&$u_0$
}

\subsection{Runge-Kutta-N (RK-N)}

\importabflex{rcl}{
$Y_1$&=&$U_n+\Delta t\sum\limits_{j=1}^s a_{1j}F(t^n+c_j\Delta t,Y_j)$\\
$Y_2$&=&$U_n+\Delta t\sum\limits_{j=1}^s a_{2j}F(t^n+c_j\Delta t,Y_j)$\\
$\vdots$&$\vdots$&$\vdots$\\
$Y_s$&=&$U_n+\Delta t\sum\limits_{j=1}^s a_{sj}F(t^n+c_j\Delta t,Y_j)$\\
$U_{n+1}$&=&$U_n+\Delta t\sum\limits_{j=1}^s b_jF(t^n+c_j\Delta t,Y_j)$\\
$U_0$&=&$u_0$
}

The Runge-Kutta Method can be represented by the Butcher Tableau.

\mportabflex{c|cccc}{
$c_1$&$a_{11}$&$a_{12}$&$\cdots$&$a_{1s}$\\
$c_2$&$a_{21}$&$a_{22}$&$\cdots$&$a_{2s}$\\
$\vdots$&$\vdots$&$\vdots$&&$\vdots$\\
$c_s$&$a_{s1}$&$a_{s2}$&$\cdots$&$a_{ss}$\\
\hline
&$b_1$&$b_2$&$\cdots$&$b_s$
}

\mportant{$\sum\limits_{j=1}^s a_{ij}=c_i\qquad \sum\limits_{j=1}^s b_j=1$}

\note{If the conditions above hold, RK is consistent.}

\mportant{$\sum\limits_{j=1}^s b_j c_j =\onha$}

\note{If the condition above holds, RK is second order accurate.}

\mportable{
$\sum\limits_{j=1}^s b_jc_j^2$&$=\frac{1}{3}$\\
$\sum\limits_{j=1}^s\sum\limits_{i=1}^s b_ia_{ij}c_j$&$=\frac{1}{6}$
}

\note{If the conditions above hold, RK is third order accurate.}

\subsubsection{Explicit RK}

\importname{$A$ strictly lower triangular.}{$a_{ij}=0\ \text{if}\ j\geq i$}

\note{If the condition above holds, RK is an explicit method.}

For explicit methods only function evaluations are made. No inverted matrices.

\mportant{$U_n=Y_1\mapsto Y_2\mapsto\cdots\mapsto Y_s\mapsto U_{n+1}$}

\note{$Y_i=f(Y_1,\ldots,Y_{i-1})$}

\subsubsection{Diagonally Implicit RK (DIRK)}

\importname{A lower triangular.}{$a_{ij}=0\ \text{if}\ j>i,\quad a_{ii}\neq 0\ \text{for some i}$}

\note{If the condition above holds, RK is diagonally implicit.}

Requieres the solution of nonlinear equations.

\mportant{$Y_i-\Delta a_{ii}F(t^n+c_i\Delta t,Y_i)=U_n+\Delta t\sum\limits_{j=1,j<i}^sa_{ij}F(t^n+c_i\Delta t,Y_i)$}

\note{$Y_i=f(Y_1,\ldots,Y_{i-1},Y_i)$}

\section{Multi-Step Methods}

\importname{linear multi-step method}{$\sum\limits_{j=0}^\gamma\alpha_jU_{n+j}=\Delta t\sum\limits_{j=0}^\gamma\beta_jF(t^{n+j},U_{n+j})$}

\note{The method is explicit if $\beta_\gamma =0$.}

\subsection{Adam's Method}

\mportabflex{r@{ = }l}{
$\alpha_\gamma$&$1$\\
$\alpha_{\gamma-1}$&$-1$\\
$\alpha_j$&$0$
}

\important{$U_{n+\gamma}=U_{n+\gamma-1}+\Delta t\sum\limits_{j=0}^\gamma\beta_jF(t^{n+j},U_{n+j}$}

If focussing on a autonomous ODE ($F(t,u(t))=F(u)$) we obtain:

\important{$U_{n+\gamma}=U_{n+\gamma-1}+\Delta t\sum\limits_{j=0}^\gamma\beta_jF(U_{n+j})$}

\subsection{Adams-Bashforth Method}

\mportant{$\beta_\gamma =0$}

\note{Thus making Adam's Method explicit.}

\important{$U_{n+\gamma}=U_{n+\gamma-1}+\Delta t\sum\limits_{j=0}^{\gamma-1}\beta_jF(U_{n+j})$}

\subsubsection{Approach to find $\beta_j$}

Try to solve: 

$u(t^{n+\gamma})-u(t^{n+\gamma-1)}=\int_{t^{n+\gamma-1}}^{t^{n+\gamma}}u'(s)ds=\int_{t^{n+\gamma-1}}^{t^{n+\gamma}}F(u(s))ds$

\finn

The integral can then be approximated using numerical quadrature rules:

\footnotesize
\importabflex{cr@{ = }l}{
$(AB1)$&$U_{n+1}$&$U_n+\Delta tF(U_n)$\\
$(AB2)$&$U_{n+2}$&$U_{n+1}+\frac{\Delta t}{2}(-F(U_n)+3F(U_{n+1})$\\
$(AB3)$&$U_{n+3}$&$U_{n+2}+\frac{\Delta t}{12}(5F(U_n)-16F(U_{n+1})+23F(U_{n+2})$
}\normalsize

\subsection{Adams-Moulton Methods}

\mportant{$\beta_\gamma\neq 0$}

\note{Thus making Adam's Method implicit.

The resulting methods are $(\gamma+1)$-order accurate.
}

\footnotesize
\importabflex{l}{
$(AM1):\ U_{n+1}=U_n+\frac{\Delta t}{2}(F(U_n)+F(U_{n+1}))$\\
$(AM2):\ U_{n+2}=U_{n+1}+\frac{\Delta t}{12}(-F(U_n)+8F(U_{n+1})+5F(U_{n+2}))$\\
$(AM3):\ U_{n+3}=U_{n+2}+\frac{\Delta t}{24}(F(U_n)-5F(U_{n+1})+19F(U_{n+2})$\\
$\hspace{20ex}+9F(U_{n+3}))$
}
\normalsize

\subsection{Truncation Error}

\importabflex{r@{ = }l}{
$T_{n+\gamma}$&$\frac{1}{\Delta t}\left(\sum\limits_{j=0}^\gamma\alpha_ju(t^{n+j})-\Delta t\sum\limits_{j=0}^\gamma\beta_jF(u(t^{n+j}))\right)$\\
&$\frac{1}{\Delta t}\left(\sum\limits_{j=0}^\gamma\alpha_ju(t^{n+j})-\Delta t\sum\limits_{j=0}^\gamma\beta_ju'(t^{n+j})\right)$
}

\note{Substitute with taylor expansions:}

\scriptsize
\mportabflex{r@{ = }l}{
$u(t^{n+j})$&$u(t^n)+j\Delta tu'(t^n)+\frac{j^2\Delta t^2}{2}u''(t^n)+..+\frac{j^k\Delta t^k}{k!}u^{(k)}(t^n)+..$\\
$u'(t^{n+j})$&$u'(t^n)+j\Delta t u''(t^n)+\frac{j^2\Delta t^2}{2}u'''(t^n)+..+\frac{j^k\Delta t^k}{k!}u^{(k+1)}(t^n)+..$
}\normalsize

\note{to end up with:}

\importabflex{rl}{
$T_{n+\gamma}=$&$\frac{1}{\Delta t}(\sum\limits_{j=0}^\gamma\alpha_j)u(t^n)+(\sum\limits_{j=0}^\gamma j(\alpha_j-\beta_j))u'(t^n)$\\
&$+\Delta t(\sum\limits_{j=0}^\gamma(\frac{j^2\alpha_j}{2}-j\beta_j))u''(t^n)+\ldots$\\
&$\Delta t^{k-1}(\sum\limits_{j=0}^\gamma(\frac{j^k\alpha_j}{k!}-\frac{j^{k-1}\beta_j}{(k-1)!}))u^{(k)(t^n)}$
}

\mportname{Consistency}{$\sum\limits_{j=0}^\gamma\alpha_j=0\qquad\sum\limits_{j=0}^\gamma j\alpha_j=\sum\limits_{j=0}^\gamma\beta_j$}

\footnotesize
\importname{Tunctuation error of order $\Delta t^k$}{$\sum\limits_{j=0}^\gamma\frac{j^q}{q!}\alpha_j=\sum\limits_{j=0}^\gamma\frac{j^{q-1}}{(q-1)!}\beta_j,\ \forall\ q\leq k+1$}\normalsize

\subsection{Starting Values}

A $\gamma$-step method requieres $\gamma$ starting values $U_0,U_1,\ldots,U_{\gamma-1}$. The initial condition from the IVP is thus used to find the other $U_i$ via a RK method.

\note{
\mportabflex{ll}{
explicit Adams-Bashfort&RK of order $\gamma-1$\\
implicit Adams-Moulton&RK of order $\gamma$
}
}

In both cases we use an RK with one order of accuracy less than the multi-step method. In the case of AB, RK has a one-step error $\mathcal{O}(\Delta t^\gamma)$ yielding a total error $(\gamma-1)\cdot \mathcal{O}(\Delta t^\gamma$ therefore the global error of the multi-step method is still $\mathcal{O}(\Delta t^\gamma)$.

\subsection{Discussion of MS Methods}

\begin{itemize}
\item \textbf{Advantage:} The right hand side Function F only needs to be evaluated once each time step. In contrast RK methods need multiple evaluations.
\item \textbf{Disadvantage:} Variable time-steps are difficult to implement. In addition they require several starting values.
\end{itemize}

\section{Stability of Numerical Methods for ODEs}

\mportabflex{r@{ = }l}{
$u'(t)$&$F(t,u(t))$\\
$u(0)$&$u_0$
}

\importname{Convergence condition}{$\lim\limits_{\underset{N\cdot\Delta t=T}{\Delta t\rightarrow 0}}U_N=u(T)$}

\note{For a one-step method, the starting value coincides with the initial condition. For multi-step methods we need the condition:}

\importname{Convergence of starting values}{$\lim\limits_{\Delta t \rightarrow 0}U_j(\Delta t)=u_0$}

\subsection{Convergence of Forward Euler for Linear ODEs}

\begin{multicols*}{2}
\mportabflex{r@{ = }l}{
$u'(t)$&$\lambda u(t)+g(t)$\\
$u(0)$&$u_0$
}

\mportabflex{r@{ = }l}{
$U_{n+1}$&$U_n+\Delta t(\lambda U_n+g(t^n))$\\
&$(1+\lambda\Delta t)U_n+\Delta t g(t^n)$\\
$U_0$&$u_0$
}
\end{multicols*}

\note{With g as a disturbance.

We want to calculate the error: $E_N=u(t^N)-U_N\quad T = N\Delta T$}

\important{$|E_N|\leq Te^{|\lambda|T}\mathcal{O}(\Delta t)\Rightarrow \lim\limits_{\Delta t\rightarrow 0}E_N\rightarrow 0$}

\note{Thus FE is convergent and first order accurate for linear ODEs.}

\subsection{Convergence of Forward Euler for Non-Linear ODEs}

\begin{multicols*}{2}
\mportabflex{r@{ = }l}{
$u'(t)$&$F(u(t))$\\
$u(0)$&$u_0$
}

\mportabflex{r@{ = }l}{
$U_{n+1}$&$U_n+\Delta tF(U_n)$\\
$U_0$&$u_0$
}
\end{multicols*}

\note{Assuming that the ODE is autonomous and the function F is Lipschitz continuous.}

\important{$||E_N||\leq Te^{LT}\cdot\mathcal{O}(\Delta t)\Rightarrow \lim\limits_{\Delta t\rightarrow 0}E_N = 0$}

\note{Thus FE is convergent for the general IVP.}

\subsection{Convergence of Consistent One-Step Methods}

\mportabflex{r@{ = }l}{
$U_{n+1}$&$U_n+\Delta t\Phi(U_n,t^n,\Delta t)$\\
$U_0$&$u_0$
}

Trunctuation of a consistent one-step methods:

\mportant{$T_n:=\frac{u(t^{n+1})-u(t^n)}{\Delta t}-\Phi(u(t^n),t^n,\Delta t)$}

\important{$||E_N||\leq Te^{LT}\cdot\mathcal{O}(\Delta t)\Rightarrow \lim\limits_{\Delta t\rightarrow 0}E_N=0$}

\note{Thus general explicit one-step methods are convergent if they are consistent.}

\finn

PLACEHOLDER 5.4 WHY CONVERGENCE IS NOT ENOUGH

\subsection{Absolute Stability}

\begin{multicols*}{2}
\mportabflex{r@{ = }l}{
$u'(t)$&$\lambda u(t)$\\
$u(0)$&$u_0$
}

\mportabflex{r@{ = }l}{
$U_{n+1}$&$(1+\lambda\Delta t)U_n$
}
\end{multicols*}

\importname{Absolute stability}{$|1+\lambda\Delta t|\leq 1\qquad|U_{n+1}|\leq|U_n|$}

\note{This makes only sense if $\lambda\leq 0$.}

\subsubsection{Absolute Stability of BE}

\mportant{$\frac{U_{n+1}-U_n}{\Delta t}\Rightarrow U_{n+1}=\frac{1}{1-\lambda\Delta t}U_n$}

\importname{Absolute stability}{$\frac{1}{|1-\lambda\Delta t|}\leq 1\Rightarrow \lambda\Delta t\in(-\infty,0]\cup[2,\infty)$}

\note{BE remains stable even for large time steps.}

\subsubsection{Absolute Stability of the Trapezoidal Rule}

\mportant{$\frac{U_{n+1}-U_n}{\Delta t}=\onha(\lambda U_n+\lambda U_{n+1}\rightarrow U_{n+1}=\frac{1+\frac{\lambda\Delta t}{2}}{1+\frac{\lambda\Delta t}{2}}\ U_n$}

\importname{Absolute stability}{$\left|\frac{1+\frac{\lambda\Delta t}{2}}{1-\frac{\lambda\Delta t}{2}}\right|\leq 1$}

\subsection{Stiff Problems}

\section*{PDE in General}

\mportant{$\mathbb{F}(x,u,\nabla u,D^2u,D^ku,\ldots)=0$}

\importname{Partial derivative}{$u_{x_i}=\frac{\partial u}{\partial x_i}\qquad u_{x_i,x_j}=\frac{\partial^2 u}{\partial x_i\partial x_j}$}

\importname{Gradient}{$\nabla u=(u_{x_1},u_{x_2},\ldots,u_{x_n})$}

\important{$D^2u=\begin{pmatrix}u_{x_1,x_1}&\ldots&u_{x_1,x_n}\\\vdots&&\vdots\\u_ {x_n,x_1}&\ldots&u_{x_n,x_n}\end{pmatrix}$}

\importname{Laplace operator}{$\Delta u=\sum\limits_{i=1}^nu_{x_i,x_i}$}

\section{The Poisson Equation}

\importname{Poisson Equation}{$-\Delta u=f$}

\importname{1D}{$-\Delta u = -u''(x)=f(x),\ \forall x\in(0,1)\quad u(0)=u(1)=0$}

\subsubsection{Derivation of the Poissonequation}

\begin{enumerate}
\ncompaq
\item Elastic body defined on domain $\Omega$
\item Find configuration $u(x)$ s.t. the deformation energy is minimal.
\item Dirichlet BC: $u|_{\partial \Omega}\equiv 0$
\item Total elastic energy: $J(u)=\onha \int_\Omega |\nabla u|^2dx-\int_\Omega uf dx$ 

\note{f: load function}.
\item Euler-Lagrange equation: 

$J'(u,v)=\lim\limits_{\tau\rightarrow 0}\frac{J(u+\tau v)-J(u)}{\tau}=0$

\note{v is direction in which we calculate the derivative.}
\item $|\nabla w|^2=\langle \nabla w,\nabla w\rangle$ and taking the limit thus losing higher order terms yields:

\importname{Euler-Lagrange Equation}{$\int_\Omega\langle\nabla u,\nabla v\rangle dx-\int_\Omega fvdx =0$}
\item Integration by parts yields: $-\int_\Omega v\Delta u dx-\int_\Omega vfdx+\int_{\partial\Omega}v\frac{\partial u}{\partial v}dx(x)=0$

\note{$\frac{\partial u}{\partial v}=\nabla u\cdot v$ $v$: unit outward vector, normal to boundary, $v\equiv 0$ on $\partial \Omega$}

$\Rightarrow \int_\Omega(-\Delta u-f)vdx=0\ \forall v$
\item \importname{Poisson equation}{$-\Delta u = f\quad u|_{\partial\Omega\equiv 0}$}
\end{enumerate}

\subsection{Poisson Equation in 1D}
\mportabflex{r@{ = }l}{
$-u''(x)$&$f(x),\quad\ \forall\ x\in(0,1)$\\
$u(0)$&$u(1)=0$
}

\begin{enumerate}
\ncompaq
\item $u'(y)=C_2+\int_0^yu''(z)dz=C_2-\int_0^yf(z)dz$
\item $u(x)=C_1+\int_0^xu'(y)dy=C_1+C_2x-\int_0^x\int_0^yf(z)dzdy$
\item $F(y)=\int_0^yf(z)dz\Rightarrow F'(y)=f(y)$
\item IBP: $\int_0^xF(y)dy=\int_0^xy'F(y)dy=xF(x)-\int_0^xyF'(y)dy$

$=\int_0^x(x-y)f(y)dy$
\item $u(x)=C_1+C_2x-\int_0^x(x-y)f(y)dy$
\item $u(x)=\int_0^1x(1-y)f(y)dy-\int_0^x(x-y)f(y)dy$

\note{$C_1=u(0)=0\qquad C_2=u(1)=\int_0^1(1-y)f(y)dy$}
\item \importname{Greens function}{$G(x,y)=\begin{cases}y(1-x)&0\leq y\leq x\\x(1-y)&x\leq y\leq 1\end{cases}$}
\item $u(x)=\int_0^1G(x,y)f(y)dy$
\end{enumerate}

\note{
\begin{itemize}
\ncompaq
\item The integral to find $u(x)$ is not always possible to evaluate exactly \dahe numerical quadrature rule.
\item Slight perturbations in the poisson equation invalidate our solution. Thus a \textbf{general form of the PE} is used often:

$-(a(x)u'(x))'+b(x)u'(x)+c(x)u(x)=f(x),\quad\forall x\in(0,1)$

$u(0)=u(1)=0$
\item Greens function representation are only available in 1D.
\end{itemize}
}

\subsection{Finite Difference Method}

\subsubsection{Discretization}

Discretise domain $[0,1]$ into N + 2 points $(N=\frac{1}{\Delta x}-1)$ by setting:

\mportant{$x_0=0,\quad x_{N+1}=1,\quad x_j=j\Delta x,\ j=1,\ldots,N$}

\mportant{$u_j\approx u(x_j)$}

\mportant{$f_j=f(x_j)$}

\important{$u'(x_j)\approx\frac{u_{j+1}-u_{j}}{\Delta x}$}

\important{$u''(x_j)\approx\frac{u_{j+1}-2u_j+u_{j-1}}{\Delta x^2}$}

\subsubsection{Finite difference scheme}

\mportant{$-u_{j+1}+2u_j-u_{j-1}=\Delta x^2 f_j,\quad \forall\ j=1,\ldots,N$}

\note{Yields a system of equations.}

\mportname{BC: $u_0=u(0)=0,\ j=1$}{$2u_1-u_2=\Delta x^2 f_1$}

\mportname{BC: $u_{N+1}=u(1)=0,\ j=N$}{$-u_{N-1}+2u_N=\Delta x^2f_N$}

\mportant{$U=[u_1,u_2,\ldots,u_N]^T\quad F=\Delta x^2[f_1,f_2,\ldots,f_N]^T$}

\mportant{$AU=F$}

\note{where $A=\begin{bmatrix}2&-1&0&\cdots&0\\-1&2&-1&\ddots&\vdots\\0&\ddots&\ddots&\ddots&0\\\vdots&\ddots&-1&2&-1\\0&\cdots&0&-1&2\end{bmatrix}\in\mathbb{R}^{N\times N}$}

\subsubsection{Numerical Results}

For any $\Delta x$ it holds that:

\important{$||u^{\Delta x}||_\infty\leq\frac{1}{8}||f^{\Delta x}||_\infty$}

\note{where $u^{\Delta x}=[u_1,u_2,\ldots,u_N],\quad f^{\Delta x}=[f_1,f_2,\ldots,f_N]$}

\important{$||E^{\Delta x}||_\infty\leq\frac{\Delta x^2}{96}\max\limits_{0\leq x\leq 1}|f''(x)|$}

\note{where $E^{\Delta x}=[E_1,\ldots,E_N]$}

\subsection{FDS for 2-D Poisson Equation}

\importabflex{r@{ = }l l}{
$-(u_{xx}(x)+u_{yy}(x))$&$f(x)$&for $x\in\Omega=(0,1)^2$\\
$u(x)$&$0$&for $x\in\delta\Omega$
}

\subsubsection{Discretization}

\mportabflex{r@{ = }l l@{$\quad$}|@{$\quad$}r@{ = }l l}{
$x_i$&$i\Delta x$&$\forall\ 1\leq i \leq N$&$y_j$&$j\Delta y$&$\forall \ 1\leq j\leq M$\\
$x_0$&$0$&$x_{N+1}=1$&$y_0$&$0$&$y_{M+1}=1$
}

\mportant{$u_{ij}\approx u(x_i,y_j)\qquad f_{ij}=f(x_i,y_j)$}

\important{$u_{xx}\approx\frac{u_{i+1,j}-2u_{i,j}+u_{i-1,j}}{\Delta x^2}\qquad u_{yy}\approx\frac{u_{i,j+1}-2u_{i,j}+u_{i,j-1}}{\Delta y^2}$}

\subsubsection{Finite Difference Scheme}

\mportant{$-\left(\frac{u_{i+1,j}-2u_{i,j}+u_{i-1,j}}{\Delta x^2}+\frac{u_{i,j+1}-2u_{i,j}+u_{i,j-1}}{\Delta y^2}\right)=f_{ij}$}

\note{
BC: $u_{0,j}=u_{N+1,j}=0\quad \forall\ j=0,1,\ldots,M+1$

BC: $u_{i,0}=u_{i,M+1}=0,\quad\forall\ i=0,1,\ldots,N+1$}

\subsubsection*{for N=M $\Leftrightarrow\ \Delta x = \Delta y$}

\mportabflex{r@{ = }r}{
$U$&$[u_{1,1},u_{2,1},\ldots,u_{N,1},u_{1,2},\ldots,u_{N,2},u_{1,N},\ldots,u_{N,N}]^T$\\
$F$&$\Delta x^2[f_{1,1},f_{2,1},\ldots,u_{N,1},u_{1,2},\ldots,u_{N,2},u_{1,N},\ldots,u_{N,N}]^T$
}

\mportant{$AU=F$}

\note{where $A=\begin{bmatrix}B&-I&0&\cdots&0\\-I&B&-I&\ddots&\vdots\\0&\ddots&\ddots&\ddots&0\\\vdots&\ddots&-I&B&-I\\0&\cdots&0&-I&B\end{bmatrix}\in\mathbb{R}^{N^2\times N^2}$ and 

$\qquad \quad\! B=\begin{bmatrix}4&-1&0&\cdots&0\\-1&4&-1&\ddots&\vdots\\0&\ddots&\ddots&\ddots&0\\\vdots&\ddots&-1&4&-1\\0&\cdots&0&-1&4\end{bmatrix}\in\mathbb{R}^{N\times N}$}

\section{FEM for 1D-Poisson}

\mportabflex{r@{ = }l}{
$-u''(x)$&$f(x)\quad \forall x\in(0,1)$\\
$u(0)$&$u(1)=0$
}

\subsection{Variational Principle}

\subsubsection*{Starting point}

The poisson equation represents the Euler-Lagrange equations corresponding to the solutions of the variational problem:

\mportant{$\min\limits_{u}J(u)\qquad J(u)=\onha\int_0^1|u'(x)|^2dx-\int_0^1u(x)f(x)dx$}

\subsubsection*{Requierements for u}

Dirichlet energy should be well defined thus:

\mportant{$\int_0^1|u'(x)|^2dx<\infty\qquad\left|\int_0^1u(x)f(x)dx\right|<\infty$}

\footnotesize
\mportant{$H_0^1([0,1]):=\left\{u:[0,1]\rightarrow\mathbb{R}:u(0)=u(1)=0\text{ and }\int_0^1|u'(x)|^2dx<\infty\right\}$}\normalsize

\note{$H_0^1([0,1])$ is the set of all functions that vanish at the boundary and have the property that the integral of the square of their derivative is bounded. \emph{Sobolev space}}

\subsubsection*{Do $H_0^1$ fulfil $\left|\int_0^1u(x)f(x)dx\right|<\infty$}

\mportant{$||u||_{H_0^1([0,1])}:=\left(\int_0^1|u'(x)|^2dx\right)^{\onha}$}

Use the Cauchy-Schwarz inequality: $|v\cdot w|\leq||v||||w||$

\mportant{$\left|\int_0^1u(x)f(x)dx\right|\leq\int_0^1|u(x)||f(x)|dx\leq\left(\int_0^1|u(x)|^2dx\right)^{\onha}\cdot\left(\int_0^1|f(x)|^2dx\right)^\onha$}

\important{$L^2([0,1]):=\left\{g:[0,1]\rightarrow\mathbb{R}:\int_0^1|g(x)|^2dx<\infty\right\}$}

\mportant{$||g||_{L^2([0,1])}=\left(\int_0^1|g(x)|^2dx\right)^\onha$}

\important{$u,f\in L^2([0,1])\Rightarrow \left|\int_0^1u(x)f(x)dx\right|<\infty$}

\subsubsection*{Constraints for well defined dirichlet energy}

\importabflex{ll}{
$u\in H_0^1([0,1])$,&$u\in L^2([0,1])$\\
$f\in L^2([0,1])$&
}

But by the \emph{Poincaré inequality}: $||u||_{L^2([0,1])}\leq||u||_{H_0^1([0,1])}$

\textbf{Precise problem formulation:}

Given $f\in L^2([0,1])$, find $u\in H_0^1([0,1])$, such that u minimises the energy functional $J(v)$ for all $v\in H_0^1([0,1])$.

\subsection{Variational Formulation}

Find $u\in H_0^1([0,1]):\ J'(u,v)=0=\lim\limits_{\tau\rightarrow 0}\frac{J(u+\tau v)-J(u)}{\tau}$

Thus u must satisfy: $\int_0^1u'(x)v'(x)dx=\int_0^1v(x)f(x)dx\quad (A)$

\note{

\begin{itemize}
\ncompaq
\item (A) is the variational formulation of the 1D Poisson equation. Also known as principle of virtual work.
\item (A) is well defined since

$\left|\int_0^1u'(x)v'(x)dx\right|\leq||u||_{H_0^1([0,1])}||v||_{H_0^1([0,1])}<\infty$

and since $f\in L^2([0,1])$, using Cauchy-Schwarz and Poincaré:

$\left|\int_0^1v(x)f(x)dx\right|\leq ||v||_{L^2([0,1])}||f||_{L^2({0,1})}$

$\leq||v||_{H_0^1([0,1])}||f||_{L^2([0,1])}<\infty$
\item If we set $u=v$ and use Cauchy-Schwarz and Poincaré:

$||u||_{H_0^1([0,1])}^2=\int_0^1|u'(x)|^2dx=\int_0^1u(x)f(x)dx$

$\leq||u||_{L^2([0,1])}||f||_{L^2([0,1])}\leq||u||_{H_0^1([0,1])}||f||_{L^2([0,1])}$

$\Rightarrow ||u||_{H_0^1([0,1])}\leq||f||_{L^2([0,1])}$

Thus we are provided with a stability estimate on the solution. The deflection of the membrane scales smaller than the applied load.
\end{itemize}


Derivation of the variational formulation:

\begin{enumerate}
\ncompaq
\item \mportant{$-u''(x)=f(x)$}
\item Multiply with test function v: $\qquad -u''(x)v(x)=f(x)v(x)$
\item Integrate: $\qquad -\int_0^1 u''(x)v(x)dx=\int_0^1 f(x)v(x)dx$
\item Look at LHS: 

$\int_0^1-u''(x)v(x) dx\overset{IBP}{=}[-u'(x)v(x)]-0^1-\int-u'(x)v'(x)=\int_0^1 u'(x)v'(x) dx$.
\item Thus: $\int_0^1 u'(x)v'(x) dx = \int_0^1 f(x)v(x) dx \quad \forall \ v\in V$
\end{enumerate}
}

\subsection{FEM Formulation}

$u\in V = H_0^1([0,1]):\quad \int_0^1u'(x)v'(x)dx=\int_0^1v(x)f(x)dx$

Notation: $(g,h)=\int_0^1g(x)h(x)dx\qquad$

\mportant{$\int_0^1u'(x)v'(x)dx=\int_0^1v(x)f(x)dx\longrightarrow (u',v')=(f,v)\quad (A)$}

\note{The function space V is infinite dimensional, FEM replaces V with a suitable finite-dimensional subspace $V^h\subseteq V$ and attempts to find $u\in\V^h$ such that (A) holds $\forall\ v\in\V^h$}

\mportant{\note{$\substack{V^h=\{w:[0,1]\rightarrow\mathbb{R}:\ w \text{ is continuous, }w(0)=w(1)=0\\ \text{and } w|_{[(j-1)h,jh]}\text{ is linear }\forall\ j\in\{1,\ldots,N+1\}\}}$}}

\subsubsection*{Discretization}

Domain $\Omega=[0,1]$, $h>0$, $N=\frac{1}{h}-1$, $N+2$ points:

$x_0=0,\quad x_{N+1}=1,\quad x_j=jh,\ \forall\ j\in(1,N)$

\note{$V^h$ is the set of all \textbf{continuous}, \textbf{piecewise linear} functions on $[0,1]$ with respect to the partition $[0,1]=\bigcup\limits_{j=1}^{N+1}[(j-1)h,jh]$.}

Basis of hat functions: $w(x)=\sum\limits_{j=1}^N w_j\phi_j(x)$

$\phi_j(x)=\begin{cases}\frac{x-x_{j-1}}{h}&x\in[x_{j-1},x_j)\\\frac{x_{j+1}-x}{h}&x\in[x_j,x_{j+1})\\0&\text{otherwise}\end{cases}$

\subsubsection*{FEM 1D Poisson variational formulation}

Find $u_h\in V^h$ such that $(u_h',v')=(f,v)\ \forall\ v\in V^h$

\subsection{Concrete Realisation of FEM}

\mportant{$v=\sum\limits_{j=1}^Nv_j\phi_j(x)$}

\mportant{\note{$(u_h',v')=(f,v)\Leftrightarrow \left(u_h',\left(\sum\limits_{j=1}^Nv_j\phi_j(x)\right)'\right)=\left(f,\sum\limits_{j=1}^Nv_j\phi_j(x)\right)$}}

By linearity: $\sum\limits_{j=1}^Nv_j(u_h',\phi_j')=\sum\limits_{j=1}^N v_j(f,\phi_j)$

\mportant{$(u_h',\phi_j')=(f,\phi_j)$ must hold for all $j=1,\ldots,N$}

And $u_h\in V^h$: $u_h=\sum\limits_{i=1}^Nu_i\phi_i(x)$

$\Rightarrow \sum\limits_{i=1}^Nu_i(\phi_i',\phi_j')=(f,\phi_j)$

\subsubsection*{Matrix formulation}

\mportant{$A=\{A_{ij}\}_{i,j=1,\ldots,N}\quad A_{ij}=(\phi_i',\phi_j')$}

\note{$U=\{u_j\}_{j=1}^N\qquad F=\{F_j\}_{j=1}^N\qquad \sum\limits_{i,j}A_{ij}u_j=F_j$}

\important{$AU=F$}

\note{A is termed the stiffness matrix, F is termed the load vector and U is termed the solution vector.}

\note{
\begin{enumerate}
\ncompaq
\item A is symmetric: $A_{ij}=(\phi_i,\phi_j)=(\phi_j,\phi_i)$
\item A is positive definite:
\item \textbf{A is invertible and FEM well defined}.
\end{enumerate}
}

\subsection{Convergence Analysis}

MISSING

\section{FEM for 2D-Poisson}

\mportant{$-\Delta u=f$ in $\Omega\qquad \left. u\right|_{\partial \Omega}\equiv 0$}

\subsection*{Definitions}

\important{$L^2(\Omega)=\left\{v:\Omega\rightarrow\mathbb{R}\int_\Omega|v|^2dx<\infty\right\}$}

\mportant{$||v||_{L^2(\Omega)}=\left(\int_\Omega|v|^2dx\right)^{1/2}$}

\importname{Inner product}{$(u,v)_{L^2(\Omega)}=\int_\Omega u(x)v(x)dx$}

\important{$H_0^1(\Omega)=\left\{v:\Omega\rightarrow:\int_\Omega|\nabla v|^2dx<\infty\right.$ and $v=0$ on $\left.\partial\Omega\right\}$}

\mportant{$||v||_{H_0^1(\Omega)}=\left(\int_\Omega|\nabla v|^2dx\right)^\onha$}

\mportant{$(u,v)_{H_0^1(\Omega)}=\int_\Omega \langle\nabla u,\nabla v\rangle dx$}

\note{$\langle \cdot,\cdot\rangle$ - Dot product in 2D}

\importname{Poincaré inequality}{$\int_\Omega |v|^2dx\leq C\int_\Omega |\nabla v|^2dx$}

\subsection{Variational Formulation}

\mportname{Dirichlet Energy}{$J(w)=\onha \int_\Omega|\nabla w|^2dx-\int_\Omega w(x)f(x) dx$}

Find $u\in V=H_0^1(\Omega)$ such that for all $v\in V$ it holds that

\mportant{$(u,v)_{H_0^1(\Omega)}=(f,v)_{L^2(\Omega)}$}

or equivalently

\mportant{$\int_\Omega \langle \nabla u,\nabla v\rangle dx  =\int_\Omega f(x)v(x) dx$}

\subsection{FEM Formulation}

$V$ has infinite dimensions. In order to discretize it we have to consider a finite-dimensional subspace $V^h$.

\subsubsection*{Triangulation}

\mportant{$\bar{\Omega}=\bigcup\limits_{K\in T_h}\bar{K}=\bar{K}_1\cup \bar{K}_2\cup\ldots\cup \bar{K}_M$}

\note{$K_i$ are non-overlapping triangles. $k_i\cap k_j = \phi =$ common vertices = common edges.

$T_h$ denotes the whole triangulation.}

\mportant{Mesh width: $h=\max\limits_{K\in T_h}\text{diam} (K)$}

\note{$\text{diam}(K)$: length of the longest edge of the triangle K

$\mathcal{N}_i$ : set of nodes (vertices)
}

\note{
\important{$V^h=\left\{v:\Omega\rightarrow\mathbb{R}:v\right.$ cont., $v|_K$ linear for each $K\in T_h$, $v=0$ on $\left.\partial\Omega\right\}$}}

\note{$V^h$ is the space of continuous, piecewise linear functions that vanish on the boundary $\partial\Omega$.}

\subsection{Concrete Realisation of FEM}

\importname{Hat functions}{$\phi_j(x)\in V^h,\quad j=1,\ldots,N$}

\note{
\mportant{where $\phi_j(\mathcal{N}_i)=\begin{cases}1&\text{if } j=1\\0&\text{otherwise}\end{cases}$}}

\importname{Basis for $V^h$}{$v(x)=\sum\limits_{j=1}^N v_j\phi_j(x)$}

\note{where $v_j=v(\mathcal{N}_j)\ \forall\ j=1,\ldots,N$}

Thus we can formulate the problem as:

\mportabflex{r@{ = }l}{
$(u_h,\phi_j)_{H_0^1(\Omega)}$&$(f,\phi_j)_{L^2(\Omega)}$\\
$\int_\Omega\langle\nabla u_h,\nabla \phi_j\rangle dx$&$\int_\Omega f(x)\phi_j(x),\quad \forall 1\leq j\leq N$\\
$u_h$&$\sum\limits_{i=1}^N u_i\phi_i$\\
$\sum\limits_{i=1}^N u_i\int_\Omega \langle\nabla\phi_i,\nabla\phi_j\rangle dx$&$\int_\Omega f(x)\phi_j(x) dx$
}

\importabflex{ll}{
$U=\{u_i\}_{i=1}^N$&$F=\{F_j\}_{j=1}^N$\\
$A=\{A_{ij}\}_{i,j=1}^N$&\\
$F_j=\int_\Omega f(x)\phi_j(x) dx$&$A_{ij}=\int_\Omega\langle\nabla \phi_i,\nabla\phi_j\rangle dx$\\
\multicolumn{2}{|c|}{$AU=F$}
}

\mportant{$A=\begin{bmatrix}B&-I&0&\cdots&0\\-I&B&-I&\ddots&\vdots\\0&\ddots&\ddots&\ddots&0\\\vdots&\ddots&-I&B&-I\\0&\cdots&0&-I&B\end{bmatrix}$}

\mportant{$B=\begin{bmatrix}4&-1&0&\cdots&0\\-1&4&-1&\ddots&\vdots\\0&\ddots&\ddots&\ddots&0\\\vdots&\ddots&-1&4&-1\\0&\cdots&0&-1&4\end{bmatrix}$}

\note{\begin{itemize}
\item A is symmetric.
\item A is positive definite.
\end{itemize}
}

\section{Implementation of the FEM}

\subsection{Step 1: Triangulation}

\mportant{Exemplary triangulation $T_h$}

\mypic{Triangulation}

\begin{center}
\note{
$Z=\begin{bmatrix}0 &  \onha & 1 & \frac{3}{2} & 0 & \onha & \frac{5}{4} & 0 & 1 & \frac{3}{2}\\ 0 & 0 & 0 & 0 & \onha & \onha & \onha & 1 & 1 & 1 \end{bmatrix}$

$ T = \begin{bmatrix} 1 & 2 & 2 & 3 & 3 & 4 & 7 & 6 & 6 & 5 \\ 2 & 5 & 3 & 6 & 4 & 7 & 9 & 7 & 8 & 6\\ 5 & 6 & 6 & 7 & 7 & 10 & 10 & 9 & 9 & 8\end{bmatrix}$
}
\end{center}

\mportant{$Z\in\mathbb{R}^{2\times N}$}

\note{
\begin{itemize}
\item $Z(\cdot ,j)$ refers to node $\mathcal{N}_j$
\item $Z(1,j)$, $Z(2,j)$ refer to x and y of $\mathcal{N}_j$
\end{itemize}
}

\mportant{$T\in\mathbb{R}^{3\times M}$}

\note{
\begin{itemize}
\item $T(\cdot, j)$ refers to the $j^{th}$ triangle $K_j$
\item $T(i,j),(i=1,2,3)$ represent the indices of the nodes of $K_j$
\end{itemize}
}

\note{Often an additional vector denoting boundary nodes is added, enabling boundary conditions.}

\subsection{Step 2: Element Stiffness Matrices and Element Load Vectors}

\mportant{$A_{ij}=\int_\Omega \langle \nabla\phi_i,\nabla\phi_j\rangle dx = \sum\limits_{m=1}^M\int_{K_m}\langle \nabla \phi_i,\nabla\phi_j\rangle dx$}

\note{$\int_{Km}\langle\nabla\phi_i,\nabla\phi_j\rangle dx \neq 0 $ only iff $\mathcal{N}_i$ and $\mathcal{N}_j$ are vertices of the same triangle!}

\mportant{$T(\alpha, m),\ \alpha=1,2,3$ - Labels of the vertices of $K_m$}

\mportant{$Z(i,T(\alpha,m)),\ i=1,2,\ \alpha = 1,2,3 $ - Coordinates of each vertex $T(\alpha,m)$}

\importname{Local shape functions}{$\phi_\alpha\left(\mathcal{N}_{T(\beta,m)}\right)=\begin{cases}1&\text{if }\alpha = \beta\\0&\text{if }\alpha\neq\beta\end{cases}$}

\mypic{ShapeFunctions}

\importname{Stiffness matrix $A^m$ for $K_m$}{$A_{\alpha,\beta}^m=\int_{K_m}\langle\nabla\phi_\alpha,\nabla\phi_\beta\rangle dx$}

\note{\begin{itemize}
\item $A^m\in\mathbb{R}^{3\times 3}$ is symmetric.
\end{itemize}
}

\importname{Element load vector $F^m$ for $K_m$}{$F_\alpha^m=\int_{K_m}f(x)\phi_\alpha(x) dx$}

\subsection*{Reference Triangle}

\mypic{ReferenceTriangle}

\mportname{Mapping}{$\Phi_K:\hat{K}\rightarrow K$}

\importable{
$x=\Phi_K(\hat{x})$&$=\begin{pmatrix}\mathcal{N}_b-\mathcal{N}_a&\mathcal{N}_c-\mathcal{N}_a\end{pmatrix}\hat{x}+\mathcal{N}_a$\\
&$=J_K\hat{x}+\mathcal{N}_a$}

\important{$F_\alpha^K=\int_Kf(x)\phi_\alpha(x)dx=\int_{\hat{K}}f(\Phi_K(\hat{x}))\hat{\phi}_\alpha(\hat{x})|\det{(J_K)}|d\hat{x}$}

\importable{$A_{\alpha,\beta}^K$&$=\int_K\langle\nabla\phi_\alpha,\nabla\phi_\beta\rangle dx$\\
&$\int_{\hat{K}}\langle J_K^{-T}\hat{\nabla}\hat{\phi}_\alpha,J_K^{-T}\hat{\nabla}\hat{\phi}_\beta\rangle\left|\det{J_K}\right|d\hat{x}$
}

\note{
\begin{itemize}
\item $\hat{\phi}_\alpha$ local shape functions of the reference triangle.
\item $\phi_\alpha(x)=\hat{\phi}_\alpha(\hat{x})$
\item $\nabla\phi_\alpha = J_K^{-T}\hat{\nabla}\hat{\phi}_\alpha (\hat{x})$
\end{itemize}
}

\subsection{Step 3: Assembly}

\begin{enumerate}
\item A = zeros(N,N);
\item f = zeros(N);
\item looping over all triangles:

$\qquad$ fetch $A^m$ and $F^m$

$\qquad$ A(T($\alpha$,m),T($\beta$,m))=A(T($\alpha$,m),T($\beta$,m))+$A^m_{\alpha,\beta}$

$\qquad$ F(T($\alpha$,m)=F($\alpha$,m)+$F_\alpha^m$

\end{enumerate}

\section{Parabolic PDE}

\mportant{$Au_{xx}+2Bu_{xt}+Cu_{tt}=F(x,t,u,u_x,u_t,\ldots)$}

\mportabflex{l@{ \dahe}l}{
$AC-B^2<0$&hyperbolic\\
$AC-B^2=0$&parabolic\\
$AC-B^2>0$&elliptic
}

\importname{Heat Equation}{\begin{tabular}{l@{=}ll}$u_t-\Delta u$&$0$&on $\Omega\times(0,T)$\\$u(x,0)$&$u_0(x)$&on $\Omega$\\$u(0,t)$&$u(1,t)=0$&on $(0,T)$\end{tabular}}

\subsection{Seperation of Variables}

\mportant{$u=u(x,t)=\mathcal{T}(t)\mathcal{X}(x)$}

\mportant{$\frac{\mathcal{T}'(t)}{\mathcal{T}(t)}=\frac{\mathcal{X}''(x)}{\mathcal{X}(x)}=-\lambda_k$}

Solve these two ODE using BC yields:

\mportabflex{l@{ = }l}{
$\mathcal{X}_k(x)$&$\sin(k\pi x),\ \lambda_k=(k\pi)^2\ \forall k\in\mathbb{Z}$\\
$\mathcal{T}(t)$&$e^{-(k\pi)^2t}$
}

\important{$u(x,t)=\mathcal{X}(x)\mathcal{T}(t)=e^{-(k\pi)^2t}\sin(k\pi x)$}

It remains to find a particular solution.

\mportant{$\int_0^1\sin(k\pi x)\sin(m\pi x)=\begin{cases}0&\text{if } k\neq m\\\frac{1}{2}&\text{if }k=m\end{cases}$}

Based on the theory of the fourier series:

\mportant{$u_0(x)=\sum\limits_{k=1}^\infty u_k^0\sin(k\pi x)$}

\note{where $u_k^0=2\int_0^1u_0(x)\sin(k\pi x)dx$}

\important{$u(x,t)=\sum\limits_{k=1}^\infty u_k^0 e^{-(k\pi)^2t}\sin(k\pi x)$}

\note{This solution fulfils the initial PDE as well as the boundary conditions.}

\subsubsection*{Evaluation of Seperation of Variables}

\begin{enumerate}
\ncompaq
\item Expand $u_0$ using the Fourier series:

$u_0^N(x)=\sum\limits_{k=1}^Nu_k^0\sin(k\pi x)$

\note{The error $|u_0-u_0^N|$ is small for large N and if $u_0$ is smooth and satisfies $u_0(0)=u_0(1)=0$}
\item Calculate the coefficients $\{u_k^0\}_{k=1}^N$

$u_k^0=2\int_0^1u_0(x)\sin(k\pi x)dx$
\item $u_N(x,t)=\sum\limits_{k=1}^Nu_k^0e^{-(k\pi)^2t}\sin(k\pi x)$

\note{\textbf{Two error sources:} Error due to the finite-truncation of the Fourier series, i.e. the series is only exact if infinite!

Error due to the use of quadrature rules to approximate integrals.
}
\end{enumerate}

\subsubsection{Application of Gauss-Green}

\importname{\\ Rate of change of $u$ in $\omega$ = Flux over the boundary + Sources - Sinks}{$\ddt\int_\omega u(x,t)dx=-\int_{\partial\omega}F\cdot vd\sigma(x)$}

\note{$u(x,t)\ \rightarrow$ some quantity (velocity, temperature, concentration, pressure,...), $F\rightarrow$ flux, $v\rightarrow$ unit outward normal}

$\int_\omega u_t dx=-\int_{\partial\omega}F\cdot v d\sigma(x)$

Gauss-Green (Integration by parts): $\int_{\partial\omega}F\cdot v d\sigma (x) = \int_\omega\nabla(F) dx$

$\int_\omega (u_t+\nabla F) dx =0\forall \omega\in\Omega\Rightarrow$ \fbox{$u_t+\nabla (F) = 0$}

\note{where $F=F(t,x,u,\nabla u,\ldots)$}

Special case: heat conduction: $F\propto -\nabla u\Rightarrow u_t=\Delta u$


\subsection{Energy Estimate}

\mportname{Energy estimate}{$\mathcal{E}(t):=\onha \int_0^1|u(x,t)|^2dx$}

\note{$\frac{d\mathcal{E}}{dt}=\onha\int_0^1(u^2)_tdx=\int_0^1uu_tdx=\int_0^1uu_{xx}dx=$

$=-\int_0^1 u_x^2dx+[uu_x]_0^1 \overset{\text{BC}}{=}-\int_0^1u_x^2dx$}

\important{$\frac{d\mathcal{E}}{dt}=-\int_0^1u_x^2dx\leq 0\Rightarrow \mathcal{E}(t)\leq \mathcal{E}(0)$}

\subsection{Uniqueness of a solution}

\note{$w:=u-\bar{u}$}

\mportabflex{r@{ = }ll}{
$w_t-w_{xx}$&$0$&on $(0,1)\times(0,T)$\\
$w(x,0)$&$0$&on $(0,1)$\\
$w(0,t)$&$w(1,t)=0$&on $(0,T)$
}

\mportant{$\bar{\mathcal{E}}(t)=\onha\int_0^1w^2(x,t)dx\Rightarrow \bar{\mathcal{E}}(t)\leq\bar{\mathcal{E}}(0)$}

\mportant{$\int_0^1 w^2(x,t)dx\leq\int_0^1w^2(x,0)dx\Rightarrow\int_0^1x^2(x,t)dx\leq 0$}

\important{$w(x,t)\equiv 0\Rightarrow u(x,t)=\bar{u}(x,t)$}

\subsection{Maximum Principles}

\important{$\max\limits_{\substack{0\leq x\leq 1,\\0\leq t\leq T}}u(x,t)\leq\max(0,\max\limits_x(u_0(x)))$}

\subsection{Explicit Finite Difference Schemes for the Heat Equation}

\begin{enumerate}
\item Discretising the Domain:

\note{$\Delta x>0\quad \delta t > 0 \quad N =\frac{1}{\Delta x}-1$

$x_0=0,\quad x_j=j\Delta x,\ j=1,\ldots, N,\quad x_{N+1}=1$

$\rightarrow N+2$ equally spaced points

$t_0=0,\quad t^n=n\Delta t,\ n=1,\ldots,M,\quad t^{M+1}=T$
}

\item Discretising the Solution u

$U^n_j\approx u(x_j,t^n)$
\item Discretising the Derivatives

\mportant{$u_{xx}(x_j,t^n)\approx \frac{U_{j+1}^n-2U_j^n+U_{j-1}^n}{\Delta x^2}$}

\mportant{$u_t(x_j,t^n)\approx\frac{U_j^{n+1}-U_j^n}{\Delta t}$}

\item The Finite Difference Scheme

\mportant{$\frac{U_j^{n+1}-U_j^n}{\Delta t}-\frac{U_{j+1}^n-2U_j^n+U_{j-1}^n}{\Delta x^2}=0$}

\note{$\lambda = \frac{\Delta t}{\Delta x^2}$}

$U_j^{n+1}=(1-2\lambda)U_j^n+\lambda_{j+1}^n+\lambda U_{j-1}^n$

Initial and boundary conditions:

\mportabflex{l@{ = }ll}{
$U_j^0$&$u(x_j,0)=u_0(x_j),$&$\forall\ 1\leq j\leq N$\\
$U_0^n$&$U_{N+1}^n\equiv 0$&$\forall\ 0\leq n\leq M+1$
}

\end{enumerate}

\note{This method is called \glqq explicit\grqq since we use the explicit euler method for time stepping.
}

\importname{Stability condition}{$\Delta t\leq \frac{1}{2}\Delta x^2$}

\subsubsection{Simplified notation}

\note{
\mportabflex{ll@{ = }l}{
Forward difference in space&$D^+_x w_j^n$&$\frac{w_{j+1}^n-w_j^n}{\Delta x}$\\
Backward difference in space&$D_x^-w_j^n$&$\frac{w_j^n-w_{j-1}^n}{\Delta x}$\\
Forward difference in time&$D_t^+w_j^n$&$\frac{w_j^{n+1}-w_j^n}{\Delta t}$\\
Backward difference in time&$D_t^-w_j^n$&$\frac{w_j^n-w_j^{n-1}}{\Delta t}$
}}

The finite difference scheme can then be recast as: 

\mportant{$D_t^+U_n^j-D_x^-D_x^+U_n^j=0$}

\subsubsection{Truncation Error}

\mportant{$\tau_j^n=\frac{u_j^{n+1}-u_j^n}{\Delta t}-\frac{u_{j-1}^n-2u_j^n+u_{j+1}^n}{\Delta x^2}$}

\note{or equivalently $\tau_j^n=D_t^+u_j^n-D_x^-D_x^+u_j^n$}

\important{$|\tau_j^n|\leq C(\Delta t+\Delta_x^2)$ or $\sqrt{\frac{\Delta x}{2}\sum\limits_{j=1}^N|u_j^n-U_j^n|^2}\leq \bar{C}(\Delta t+\Delta x^2)$}

\note{Thus the explicit finite difference scheme has a first-order rate of convergence in time and a second-order rate of convergence in space.}

\subsection{An Implicit Finite Difference Scheme}

\mportant{$D_t^-U_j^{n+1}=D_x^-D_x^+U_j^{N+1}$}

\mportant{$\frac{U_j^{n+1}-U_j^n}{\Delta t}=\frac{U_{j-1}^{n+1}-2U_j^{n+1}+U_{j+1}^{n+1}}{\Delta x^2}$}

\note{$\lambda = \frac{\Delta t}{\Delta x^2}$}

\mportant{$-\lambda U_{j-1}^{n+1}+(1+2\lambda)U_j^{n+1}-\lambda U_{j+1}^{n+1}=U_j^n$}

\mportant{$A U^{n+1} = F^n$}

\important{$U^{n+1}=\{U_j^{n+1}\}_{j=1}^N\qquad F^n=\{U_j^n\}_{j=1}^N$}

\important{$A =\begin{bmatrix}1+2\lambda&-\lambda&0&\cdot&0\\-\lambda&1+2\lambda&-\lambda&\cdot&\cdot\\0&\cdot&\cdot&\cdot&0\\ \cdot&\\\cdot&-\lambda&1+2\lambda&-\lambda\\0&\cdot&0&-\lambda&1+2\lambda\end{bmatrix}$}

\important{Unconditional stability!}

\subsection{Crank-Nicloson Scheme}

\mportant{$D_t^+U_j^n=\onha D_x^-D_x^+U_j^n+\onha D_x^2D_x^+U_j^{n+1}$}

\mportant{$\frac{U_j^{n+1}-U_j^n}{\Delta t}=\frac{U_{j-1}^n-2U_j^n+U_{j+1}^n}{2\Delta x^2}+\frac{U_{j-1}^{n+1}-2U_j^{n+1}+U_{j+1}^{n+1}}{2\Delta x^2}$}

\note{Crank-Nicholson is the formal average of the explicit finite difference scheme and the implicit finite difference scheme.}

\mportant{$U_0^n=U_{N+1}^n=0\qquad U_j^0=u_j^0=u_0(x_j)\qquad \lambda = \frac{\Delta t}{\Delta x^2}$}

\mportant{$-\frac{\lambda}{2}U_{j-1}^{n+1}+(1+\lambda)U_j^{n+1}-\frac{\lambda}{2}U_{j+1}^{n+1}=\frac{\lambda}{2}U_{j-1}^n+(1-\lambda)U_j^n-\frac{\lambda}{2}U_{j+1}^n=F_j^n$}

\mportant{$AU^{n+1}=F^n$}

\important{$U^{n+1}=\{U_j^{n+1}\}_{j=1}^N\qquad F^n=\{F_j^n\}_{j=1}^N$}

\important{$A=\begin{bmatrix}1+\lambda&-\frac{\lambda}{2}&0&\cdot&0\\-\frac{\lambda}{2}&1+\lambda&-\frac{\lambda}{2}&\cdot&\cdot\\0&\cdot&\cdot&\cdot&0\\ \cdot&\cdot&-\frac{\lambda}{2}&1+\lambda&-\frac{\lambda}{2}\\0&\cdot& 0&-\frac{\lambda}{2}&1+\lambda\end{bmatrix}$}

\important{Unconditionally stable!}

\note{Despite the unconditional stability the CN-scheme \textbf{only satisfies the discrete maximum principle if} $\lambda = \frac{\Delta t}{\Delta x^2}\leq 1$. For large values of $\lambda$ the solution may contain spurious oscillations.}

\subsubsection{Truncation error}

\mportant{$\tau_j^n=\frac{u_j^{n+1}-u_j^n}{\Delta t}-\frac{u_{j-1}^n-2u_j^n+u_{j+1}^n}{2\Delta x^2}-\frac{u_{j-1}^{n+1}-2u_j^{n+1}+u_{j+1}^{n+1}}{2\Delta x^2}$}

\mportant{$\tau_j^n=D_t^+u_j^n-\left(\onha D_x^-D_x^+u_j^n+\onha D_x^-D_x^+u_j^{n+1}\right)$}

\important{$|\tau_j^n|\leq C(\Delta t^2+\Delta x^2)$}

\note{CN has a second-order rate of convergence in both time and space.}

\subsection{Convergence Studies}


\section{Linear Transport Equations (Hyperbolic PDEs)}

\mportant{$u_t+a(x,t)u_x = 0\ \forall (x,t)\in\mathbb{R}\times\mathbb{R}_+$}

\subsection{Method of Characteristics}

Assume that we have a curve along which the solution u is constant.

\mportant{$u(x,0)=u_0(x)$}

\mportabflex{l@{ = }l}{
$0$&$\ddt u(x(t),t)$\\
&$u_t(x(t),t)+u_x(x(t),t)x'(t)$
}

\note{where x(t) is a characteristic curve.}

Since this has to fulfil the linear transport equation we know that:

\mportabflex{lll@{=}l}{
\multirow{2}{*}{characteristics}&\ldelim\{{2}{1pt}&$x'(t)$&$a(x(t),t)$\\
&&$x(0)$&$x_0$
}

\subsubsection{Gronwall's inequality}

\mportant{$u'(t)\leq\beta(t)u(t)\quad\forall\ t\in(a,b)$}

\note{where $\beta(t)$ continuous, $u(t)$ differentiable on some $[a,b]$.}

\important{$u(t)\leq u(a) \exp\left(\int_a^t\beta(t)dt\right)\quad\forall\ t\in[a,b]$}

Let $u(x,t)$ be a smooth solution, $\lim\limits_{|x|\rightarrow\infty}u(x,t)=0$. Then u fulfils the following energy bound:

\important{$\int_\mathbb{R}u^2(x,t)dx\leq e^{||a||_{C^1}t}\int_\mathbb{R}u_0^2(x)dx$}

\subsection{Finite difference schemes}

\subsubsection{Discretization}

$[x_l,x_r]$ is discretized with a mesh size $\Delta x$ into a sequence of $N+1$ points.

Divide $[0,T]$ into M points $t^n=n\Delta t$

Then a finite difference scheme is applied:

\mportant{$\frac{U_j^{n+1}-U_j^n}{\Delta t}+\frac{a(U_{j+1}^n-U_{j-1}^n)}{2\Delta x}=0$}

By an energy analysis we conclude that this central difference scheme does not produce valid but oscillatory solutions with increasing energies.

\subsubsection{Upwind schemes}

The central scheme does not respect the direction of propagation of information for the transport equation. This is corrected by the upwind scheme.

As found by the method of characteristics the characteristic curves are defined by:
 
\mportant{$x'(t)=a$}

Thus if $a>0$ the direction of information propagation is from left to right, then we use a backward difference in space to obtain our scheme. We do the opposite if $a<0$.

\mportant{$a^+=\max\{a,0\},\quad a^-=\min\{a,0\},\quad |a|=a^+-a^-$}

Thus the combined (backwards/forwards) upwind scheme can be written as:

\important{$\frac{U-j^{n+1}-U_j^n}{\Delta t}+\frac{a^+(U_j^n-U_{j-1}^n)}{\Delta x}+\frac{a^-(U_{j+}^n-U_j^n)}{\Delta x}=0$}

or

\important{$\frac{U_j^{n+1}-U_j^n}{\Delta t}+\frac{a(U_{j+1}^n-U_{j-1}^n)}{2\Delta x}=\underbrace{\frac{|a|}{2\Delta x}(U_{j+1}^n-2U_j^n+U_{j-1}^n)}_\text{numerical viscosity}$}

However the upwind scheme still is only stable for some $\frac{\Delta t}{\Delta x}.$

\subsubsection{Stability for the upwind scheme}

If $|a|\frac{\Delta t}{\Delta x}\leq 1$ the upwind scheme satisfies the energy estimate $E^{n+1}\leq E^n$ and is thus \textbf{conditionally stable.}










%\section{Linear Transport Equations}

\end{multicols*}

\end{document}










































